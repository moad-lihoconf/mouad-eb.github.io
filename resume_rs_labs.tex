\documentclass[11pt,a4paper]{article}

% Packages
\usepackage[utf8]{inputenc}
\usepackage[T1]{fontenc}
\usepackage[margin=0.6in, top=0.55in, bottom=0.55in]{geometry}
\usepackage{titlesec}
\usepackage{enumitem}
\usepackage{hyperref}
\usepackage{xcolor}
\usepackage{tabularx}
\usepackage{parskip}

% Colors
\definecolor{primarycolor}{RGB}{0, 51, 102}
\definecolor{linkcolor}{RGB}{0, 82, 155}
\definecolor{darkgray}{RGB}{55, 55, 55}

% Hyperlink setup
\hypersetup{
    colorlinks=true,
    linkcolor=linkcolor,
    urlcolor=linkcolor,
    pdfauthor={Mouad El Bouchattaoui},
    pdftitle={Mouad El Bouchattaoui - Academic CV}
}

% Section formatting - clean line underneath
\titleformat{\section}
    {\large\bfseries\color{primarycolor}\scshape}
    {}
    {0em}
    {}
    [\vspace{-0.5em}\textcolor{primarycolor}{\rule{\textwidth}{0.8pt}}]

\titlespacing*{\section}{0pt}{10pt}{6pt}

% Remove page numbers
\pagestyle{empty}

% Custom bullet style - consistent filled circles
\setlist[itemize,1]{label=\textcolor{darkgray}{\textbullet}, leftmargin=15pt, itemsep=1pt, parsep=0pt, topsep=2pt}
\setlist[itemize,2]{label=\textcolor{darkgray}{\textbullet}, leftmargin=13pt, itemsep=1pt, parsep=0pt, topsep=1pt}

% Custom commands
\newcommand{\resumeSubheading}[4]{
    \vspace{5pt}
    \begin{tabularx}{\textwidth}{@{}X r@{}}
        \textbf{\large #1} & \textbf{#2} \\
        \textit{#3} & \textit{#4} \\
    \end{tabularx}
    \vspace{1pt}
}

\newcommand{\publicationEntry}[4]{
    \vspace{4pt}
    \noindent\textbf{#1} \hfill \textbf{#2} \\
    \textit{#3} \\
    {\small #4}
    \vspace{2pt}
}

\begin{document}

%----------HEADER----------
\begin{center}
    {\Huge\bfseries\color{primarycolor} Mouad El Bouchattaoui} \\[5pt]
    {\large\color{darkgray} Ph.D. in Applied Mathematics \;|\; Machine Learning and Causal Inference} \\[6pt]
    {\small\color{darkgray}
    \textbf{Research focus:} causal inference; treatment-effect estimation; latent-variable models; identifiability; causal representation learning; longitudinal and panel data; probabilistic modeling.
    }\\[6pt]
    {\small
    +33\,7\,55\,09\,28\,11 \;\;$\vert$\;\;
    \href{mailto:mouad.elbouchattaoui@gmail.com}{mouad.elbouchattaoui@gmail.com} \;\; $\bullet$ \;\;
    Paris Area, France \\[4pt]
    \href{https://www.linkedin.com/in/mouad-elbouchattaoui/}{LinkedIn}
    \;\; $\bullet$ \;\;
    \href{https://github.com/moad-lihoconf}{GitHub}
    \;\; $\bullet$ \;\;
    \href{https://scholar.google.com/citations?hl=fr&user=yH0gpPoAAAAJ}{Google Scholar}
    }
\end{center}

\vspace{-6pt}

%----------EDUCATION----------
\section{Education}
\resumeSubheading
    {Paris-Saclay University}{Paris-Saclay, France}
    {Ph.D. in Applied Mathematics (defended Nov.\ 2025)}{09/2021 -- 11/2025}
\begin{itemize}
    \item Dissertation: \textit{Learning causality for longitudinal data} (\href{https://arxiv.org/abs/2512.04980}{arXiv:2512.04980}).
\end{itemize}

\resumeSubheading
    {CentraleSup\'elec}{Paris-Saclay, France}
    {Engineering/Master's degree in Applied Mathematics}{09/2019 -- 11/2021}
\begin{itemize}
    \item Major in Data Science (research track).
\end{itemize}

\resumeSubheading
    {Centrale Casablanca}{Casablanca, Morocco}
    {Undergraduate Studies}{09/2017 -- 08/2019}
\begin{itemize}
    \item General engineering curriculum; research-focused track in mathematics.
\end{itemize}

%----------PUBLICATIONS----------
\section{Publications}

\publicationEntry{NeurIPS 2024}{\href{https://openreview.net/forum?id=bKOZYBJE4Z}{OpenReview}}{Causal Contrastive Learning for Counterfactual Regression Over Time}{Bouchattaoui, M. E., Tami, M., Lepetit, B., \& Courn\`ede, P. H.}

We address counterfactual regression over long time horizons while maintaining computational efficiency. Our approach leverages contrastive regularization and information-theoretic principles to capture long-term dependencies, mitigate selection bias in longitudinal data, and enable fast test-time inference.

\vspace{6pt}

\publicationEntry{TMLR 2025}{\href{https://openreview.net/forum?id=atf9q49DeF}{OpenReview}}{Causal Dynamic Variational Autoencoder for Counterfactual Regression in Longitudinal Data}{Bouchattaoui, M. E., Tami, M., Lepetit, B., \& Courn\`ede, P. H.}

We introduce CDVAE to infer unobserved risk factors affecting outcome sequences. CDVAE provides identifiability results and generalization bounds for individual treatment-effect estimation, with strong empirical performance on synthetic and real-world datasets.

\vspace{6pt}

\publicationEntry{Causal@UAI 2024 (Workshop Poster)}{\href{https://openreview.net/forum?id=dXwT6Dyo6z}{OpenReview}}{Toward a More Transparent Causal Representation Learning}{Bouchattaoui, M. E., Tami, M., Lepetit, B., \& Courn\`ede, P.-H.}

This work studies transparency and interpretability of learned causal factors in high-dimensional, time-varying data via structural sparsity in generative functions. We introduce a sparse subspace clustering technique based on decoder Jacobians to trace relationships between latent variables and observed features.

\vspace{6pt}

\publicationEntry{French Statistical Society (JdS) --- 54th Annual Meeting}{\href{https://jds2023.sciencesconf.org/resource/page/id/19}{Paper}}{A Non-Linear Mixed-Effect Model for the Estimation of Causal Effects Through Time}{Bouchattaoui, M. E., Tami, M., Lepetit, B., \& Courn\`ede, P.-H.}

We connect unobserved risk factors in causal inference with unobserved heterogeneity in mixed-effects panel models, leveraging deep non-linear mixed-effects modeling to estimate individual treatment effects over time.

%----------PREPRINTS----------
\section{Preprints (Technical Notes)}

\publicationEntry{arXiv}{\href{https://arxiv.org/abs/2407.04758}{arXiv:2407.04758}}{Random Walks in Random Environments: A Short Introduction}{Bouchattaoui, M. E.}

Expository note on random walks in random environments (RWRE), covering recurrence/localization and the electrical-network analogy, with brief applications.

\vspace{6pt}

\publicationEntry{arXiv}{\href{https://arxiv.org/abs/2407.04189}{arXiv:2407.04189}}{Meta-Learning and Representation Learning: A Short Theoretical Note}{Bouchattaoui, M. E.}

Short theoretical note formalizing meta-learning as leveraging shared structure across tasks for data-scarce adaptation.

%----------COLLOQUIA AND PRESENTATIONS----------
\section{Colloquia and Presentations}
\resumeSubheading
    {Research Seminar on Causal Representation Learning, IBM Research}{Paris-Saclay, France}
    {Toward Transparent Causal Representation Learning}{2025}

\resumeSubheading
    {Causality Reading Group, Department of Computer Science, University of Pisa}{Pisa, Italy}
    {Causal Contrastive Learning for Counterfactual Regression Over Time}{2025}

\resumeSubheading
    {NeurIPS@Paris 2024}{Paris, France}
    {Causal Contrastive Learning for Counterfactual Regression Over Time \href{https://neuripsinparis.github.io/neurips2024paris/}{(link)}}{12/2024}

\resumeSubheading
    {Workshop on \textit{Mathematical Foundations of AI} (4th edition)}{Paris, France}
    {Random Effects in Longitudinal Data Through the Lens of Causal Inference \href{https://www.dataia.eu/en/events/workshop-mathematical-foundations-ai-4th-edition}{(link)}}{09/2024}

\resumeSubheading
    {Quarter on Causality Colloquium}{Paris \& Grenoble, France}
    {CDVAE: Estimating Causal Effects Over Time Under Unobserved Adjustment Variables \href{https://quarter-on-causality.github.io/analysis}{(link)}}{2023}

%----------AWARDS----------
\section{Awards}

\begin{tabularx}{\textwidth}{@{}X r@{}}
    \textbf{Excellence Scholarship from the French Embassy} & 2019--2020 \\
    \multicolumn{2}{@{}l@{}}{\small Double Engineering Degree Program} \\[6pt]
    \textbf{First rank at Kaggle Challenge (Dreem startup, Paris)} & 2020 \\
    \multicolumn{2}{@{}l@{}}{\small Classification of sleep stages from brain signals} \\
\end{tabularx}

%----------TEACHING EXPERIENCE----------
\section{Teaching Experience}
\resumeSubheading
{CentraleSup\'elec, Paris-Saclay University}{Paris-Saclay, France}
{Assistant Teaching Professor}{2022 -- 2024}
\begin{itemize}
    \item Taught and led lab/tutorial sessions for master's-level students across multiple programs, including the \textit{Master in Data Science \& Business Analytics} (ESSEC / CentraleSup\'elec) and engineering master's programs at CentraleSup\'elec.
    \item Led lab/tutorial sessions for master's courses \textit{Introduction to Machine Learning} and \textit{Ensemble Learning}: dimensionality reduction; supervised learning; bagging; random forests; boosting; unsupervised and semi-supervised learning.
    \item Delivered a guest lecture on \textit{Introduction to Causality}, presenting core causal-inference concepts.
    \item Graded exams and served on project juries, assessing student work and providing detailed feedback.
\end{itemize}

%----------MENTORSHIP AND EXTRACURRICULAR ACTIVITIES----------
\section{Mentorship and Extracurricular Activities}

\resumeSubheading
    {CentraleSup\'elec --- MICS Lab}{Paris-Saclay, France}
    {Research Internship Mentor}{2024}
\begin{itemize}
    \item Mentored Malek Ben Salah (\'Ecole Polytechnique / ENSTA) on \textit{individual treatment-effect estimation for sequences of treatments} in metastatic breast cancer.
\end{itemize}

\resumeSubheading
    {Study Week on Causal Inference for Industry}{\href{https://quarter-on-causality.github.io/seme/\#program}{Program link}}
    {Participant \& Mentor --- Causal Discovery Hackathon}{2023}
\begin{itemize}
    \item Proposed and helped design a causal discovery challenge using real transactional data from Saint-Gobain.
    \item Mentored participating teams during the hackathon, advising on causal discovery framing, assumptions, and evaluation.
\end{itemize}

\resumeSubheading
    {Saint-Gobain}{Paris-La D\'efense, France}
    {Co-supervisor, Master's thesis}{2022}
\begin{itemize}
    \item Co-supervised Wiam Adnan (\'Ecole Centrale Marseille) on \textit{statistical impact estimation of marketing campaigns}.
\end{itemize}

%----------INDUSTRIAL RESEARCH EXPERIENCE----------
\section{Industrial Research Experience}

\begin{itemize}[leftmargin=0pt, label={}]

    \item[]
    \resumeSubheading
        {Saint-Gobain}{Paris-La D\'efense}
        {Senior Data Scientist}{12/2024 -- Present}

    \begin{itemize}
        \item Led two applied-ML projects with an emphasis on evaluation rigor and decision stability: (i) LLM-based document compliance, (ii) Bayesian predictive maintenance and root-cause analysis.
        \item \textbf{LLM-based contract compliance.} Built clause-level representations and retrieval-based evidence linking (PostgreSQL/pgvector) for grounded checklist verification; improved checklist-to-evidence stability from \textbf{$\sim$10\%} to \textbf{<1\%} across reruns via hybrid similarity tie-breaking (semantic + TF--IDF) and caching.
        \item \textbf{Predictive maintenance \& RCA.} Harmonized machine stops, quality/scrap metrics, and maintenance logs into a validated RCA dataset across \textbf{3 sites / 5 lines} with \textbf{98\%} cross-warehouse matching; trained Bayesian failure-risk models in Pyro and evaluated calibration and top-$k$ triage on historical incidents.
        \item \textit{Selected tools:} Python, SQL (PostgreSQL/pgvector, Snowflake), FastAPI, PyTorch/Pyro.
    \end{itemize}

    \item[]
    \resumeSubheading
        {Saint-Gobain}{Paris-La D\'efense}
        {Industrial PhD Researcher (Causal Inference / ML)}{09/2021 -- 12/2024}

    \begin{itemize}
        \item Industrial PhD on causal inference and sequential decisioning for commercial targeting on longitudinal customer data; resulted in publications at \textbf{NeurIPS} and \textbf{TMLR} and a workshop poster at \textbf{UAI} (see Publications).
        \item \textbf{Uplift / ITE.} Built causal uplift pipelines for engagement outcomes across \textbf{100+ campaigns}; developed neural embeddings for a \textbf{100K+}-dimensional customer panel, reducing processing time by \textbf{25\%}.
        \item \textbf{Sequential decision optimization.} Designed a sequential causal policy-learning framework to recommend multi-week action sequences using counterfactual outcome trajectories; transferred personalized recommendations to \textbf{100+ sales engineers} via decision-support tooling.
        \item \textbf{Causal representation learning.} Learned time-varying embeddings from longitudinal purchases (\textbf{1M+ transactions}) to capture evolving customer state and heterogeneity; packaged artifacts for reuse in segmentation and forecasting.
        \item \textit{Selected tools:} PyTorch/Lightning, EconML/DoubleML, MLflow, Hydra, Optuna.
    \end{itemize}

    \item[]
    \resumeSubheading
        {Saint-Gobain}{Paris-La D\'efense}
        {Data Scientist (Segmentation \& Targeting)}{04/2021 -- 08/2021}

    \begin{itemize}
        \item Built an unsupervised segmentation pipeline for commercial targeting (feature engineering + feature selection + clustering).
        \item Reduced end-to-end pipeline runtime by \textbf{85\%} and improved downstream engagement by \textbf{40\%} (click-through).
        \item Validated segment stability using distributional diagnostics (KDE, Wasserstein distance) and low-dimensional structure (UMAP).
    \end{itemize}

    \item[]
    \resumeSubheading
        {EDF Lab}{Paris-Saclay}
        {Research Intern (Time Series Representation Learning)}{05/2020 -- 11/2020}

    \begin{itemize}
        \item Developed representation learning models (AEs, VAEs, GANs) for electricity-consumption time series, improving downstream classification accuracy by \textbf{5\%} over baselines (CNN/LSTM).
        \item Optimized architectures and data pipelines, achieving \textbf{10$\times$} runtime reduction.
        \item Benchmarked deep and classical baselines to quantify accuracy--compute trade-offs.
    \end{itemize}

    \item[]
    \resumeSubheading
        {L'Or\'eal Paris}{Paris}
        {Research Intern (Time Series Analysis \& Forecasting)}{10/2019 -- 03/2020}

    \begin{itemize}
        \item Established a baseline to assess whether product-to-tester transformations are predictable versus chaotic.
        \item Engineered time-series features (tsfresh) and behavior-based clustering to uncover recurring patterns across SKUs.
        \item Compared forecasting and similarity approaches (ARIMA/SARIMA, LSTM, DeepAR, DTW) and delivered recommendations for operational forecasting.
    \end{itemize}

\end{itemize}

\end{document}

